
%%%%%%%%%%%%%%%%%%%%%%% file typeinst.tex %%%%%%%%%%%%%%%%%%%%%%%%%
%
% This is the LaTeX source for the instructions to authors using
% the LaTeX document class 'llncs.cls' for contributions to
% the Lecture Notes in Computer Sciences series.
% http://www.springer.com/lncs       Springer Heidelberg 2006/05/04
%
% It may be used as a template for your own input - copy it
% to a new file with a new name and use it as the basis
% for your article.
%
% NB: the document class 'llncs' has its own and detailed documentation, see
% ftp://ftp.springer.de/data/pubftp/pub/tex/latex/llncs/latex2e/llncsdoc.pdf
%
%%%%%%%%%%%%%%%%%%%%%%%%%%%%%%%%%%%%%%%%%%%%%%%%%%%%%%%%%%%%%%%%%%%


\documentclass[runningheads]{llncs}

\usepackage{amssymb}
\setcounter{tocdepth}{3}
\usepackage{graphicx}

\usepackage{url}
\urldef{\mailsa}\path|{g1, g2, g3, g4}@cs.stonybrook.edu|
\newcommand{\keywords}[1]{\par\addvspace\baselineskip
\noindent\keywordname\enspace\ignorespaces#1}
\newcommand{\swallow}[1]{ }

\begin{document}

\mainmatter  % start of an individual contribution

% first the title is needed
\title{Project Report: Team X\\
Name of your Challenge Here}

% a short form should be given in case it is too long for the running head
\titlerunning{Name of your Challenge Here}

% the name(s) of the author(s) follow(s) next
%
% NB: Chinese authors should write their first names(s) in front of
% their surnames. This ensures that the names appear correctly in
% the running heads and the author index.
%
\author{Group Member1 \and Group Member2 \and Group Member3 \and Group Member4}
%
\authorrunning{Member1 \and Member2 \and Member3 \and Member4}
% (feature abused for this document to repeat the title also on left hand pages)

% the affiliations are given next; don't give your e-mail address
% unless you accept that it will be published
\institute{Department of Computer Science, Stony Brook University,\\
Stony Brook, NY 11794-4400\\
\mailsa\\
\url{http://www.cs.stonybrook.edu/~skiena/591/projects}}

%
% NB: a more complex sample for affiliations and the mapping to the
% corresponding authors can be found in the file "llncs.dem"
% (search for the string "\mainmatter" where a contribution starts).
% "llncs.dem" accompanies the document class "llncs.cls".
%

\toctitle{Lecture Notes in Computer Science}
\tocauthor{Authors' Instructions}
\maketitle

\swallow{   % DO NOT BOTHER WITH THIS
\begin{abstract}
The abstract should summarize the contents of the paper and should
contain at least 70 and at most 150 words. It should be written using the
\emph{abstract} environment.
\keywords{We would like to encourage you to list your keywords within
the abstract section}
\end{abstract}
}


\section{Challenge} 

State/define the challenge you have been assigned (half page)

\section{History/Background}
Summarize the relevant information to properly understand the challenge domain (2-4 pages)
\section{Literature Review}
Perform an extensive search for sources in the academic literature as well as in the popular literature (web/books).  What issues associated with your domain have been under academic study, even if only tangentially related to your specific challenge?  (3-4 pages)
\section{Data Sets}

Do an extensive search for relevant data sets to help you understand the challenge, and/or propose your own ways to get the relevant data. Report on the size (number of rows) and composition (descriptions of the columns) of the data matrices you have assembled to build your models from.  Where did the data come from, and how satisfied are you with what you have for your project? (2-4 pages)



\section{Observations}

Report on interesting things you can learn/visualize from your data set.  Include and describe these visualizations.  (2-4 pages)


\section{Baseline Model}

Propose a trivially simple baseline model which is capable of making a prediction responding to your challenge.   Experimentally validate what the performance of your baseline model is.   (1-2 pages)

\section{Advanced Model}
Describe your development of one or more advanced models (presumably machine learning-based), and present results from your evaluation environment showing the performance of it against your baseline models. What methods of fitting/data analysis are you using? (3-5 pages)

\section{Final Prediction and Conclusions}

State your final forecast for your challenge.  Discuss any difficulties you had to overcome in building a good model, and fruitful investigations for subsequent groups.   (1-3 pages)




\subsubsection*{Acknowledgments.} Here acknowledge any other people who helped with this project.

\section{Bibliography}\label{references}

The correct BibTeX entries for the Lecture Notes in Computer Science
volumes can be found at the following Website shortly after the
publication of the book:
\url{http://www.informatik.uni-trier.de/~ley/db/journals/lncs.html}

For citations in the text please use
square brackets and consecutive numbers: \cite{jour}, \cite{lncschap},
\cite{proceeding1} -- provided automatically
by \LaTeX 's \verb|\cite| \dots\verb|\bibitem| mechanism.

Please base your references on the
examples below. 
The following section shows a sample reference list with entries for
journal articles \cite{jour}, an LNCS chapter \cite{lncschap}, a book
\cite{book}, proceedings without editors \cite{proceeding1} and
\cite{proceeding2}, as well as a URL \cite{url}.
Please note that proceedings published in LNCS are not cited with their
full titles, but with their acronyms!

\begin{thebibliography}{4}

\bibitem{jour} Smith, T.F., Waterman, M.S.: Identification of Common Molecular
Subsequences. J. Mol. Biol. 147, 195--197 (1981)

\bibitem{lncschap} May, P., Ehrlich, H.C., Steinke, T.: ZIB Structure Prediction Pipeline:
Composing a Complex Biological Workflow through Web Services. In: Nagel,
W.E., Walter, W.V., Lehner, W. (eds.) Euro-Par 2006. LNCS, vol. 4128,
pp. 1148--1158. Springer, Heidelberg (2006)

\bibitem{book} Foster, I., Kesselman, C.: The Grid: Blueprint for a New Computing
Infrastructure. Morgan Kaufmann, San Francisco (1999)

\bibitem{proceeding1} Czajkowski, K., Fitzgerald, S., Foster, I., Kesselman, C.: Grid
Information Services for Distributed Resource Sharing. In: 10th IEEE
International Symposium on High Performance Distributed Computing, pp.
181--184. IEEE Press, New York (2001)

\bibitem{proceeding2} Foster, I., Kesselman, C., Nick, J., Tuecke, S.: The Physiology of the
Grid: an Open Grid Services Architecture for Distributed Systems
Integration. Technical report, Global Grid Forum (2002)

\bibitem{url} National Center for Biotechnology Information, \url{http://www.ncbi.nlm.nih.gov}

\end{thebibliography}



\section{Checklist of Items to be Submitted for Project Final Report}
Here is a checklist of everything the volume editor requires from you:


\begin{itemize}
\settowidth{\leftmargin}{{\Large$\square$}}\advance\leftmargin\labelsep
\itemsep8pt\relax
\renewcommand\labelitemi{{\lower1.5pt\hbox{\Large$\square$}}}

\item The final \LaTeX{} source files
\item A final PDF file
\item A video file with 15-20 minutes footage since the status reel.
\item A link to this file up loaded to YouTube.
\item The powerpoint file you created for your presentation.
\item A website presenting the entire data and report for the project.
\end{itemize}
\end{document}
